\documentclass{article}
\usepackage[utf8]{inputenc}

\title{Elaboration I: Planning}
\author{Sophie Wallace }
\date{August 2019}

\begin{document}

\maketitle

\tableofcontents
\clearpage

\section{Task Outline}
Through computational analysis I was able to identify the individual processes required for my proof of concept (POC). By recognizing repetitive patterns in the second scoping exercise, I was able to create an algorithm design. This provided the initial and most tangible conceptualization of an achievable POC so far, however further elaboration of the tools, techniques and technologies involved has to be unpacked.

\section{Scoping II: Computational Analysis}
From computation analysis I was able to identify key processes necessary for my POC. From these processes, criteria for existing technologies that will be investigated include:

\begin{itemize}
\item Compatible with mobile and desktop
\item Compatible with all ethnographic data types i.e. text, photos, video, audio.
\item Organises files with rich metadata
\item Ability to edit files
\item Offline function 
\item Safe and reliable storage system
\end{itemize}


\section{Metadata Protocols}
According to the FAIR Guiding Principles for scientific data management and stewardship, the infrastructure of scholarly data needs to be urgently improved to support reputable data (Wilkinson et al 2016). FAIR standards require data that should be findable, accessible, interoperable and resusable. This ensures that data is described with rich metadata, easily identified, accessible, has authorisation where necessary, uses plain language for metadata and metadata meets domain-relevant community standards (Wilkinson et al 2016). These protocols are essential to guide and articulate the tests needed for elaboration. 

\clearpage

\section{Routes for Success}
There are a range of different pathways to achieve this POC, however tests need to be run to identify the pains and gains present in the current relevant technologies. There are two technological approaches that I will run tests with. 
\subsection{Route 1: Integrated Programs}

\begin{itemize}
\item \textbf{Open Semantic Search}
\item \textbf{Tropy}
\end{itemize}
These integrated programs are already established file management systems with a variety of functions to assist with data management and storage. Although already integrated, tests are necessary to identify the pains and gains when using these programs and whether they adhere to the desired and specified criteria.

\subsection{Route 2: Old School}
\begin{itemize}
\item \textbf{Excel + Cloudstor Sync Client}
\end{itemize}
This old school approach combines two well-known and accepted tools that work to efficiently record and store metadata. Similar tests will be run to outline whether this combined tool meets the specified criteria of this POC.

\section{Tests for Risk Assessment}
In accordance with the FAIR principles and the criteria needed for this POC to be successfully elaborated and developed, an array of tests will be need to be run. Questions will include:
\begin{itemize}
\item Is this tool compatible with both mobile and desktop?
\item Is this tool compatible with all data types?
\item Does this tool organise files with rich metadata?
\item Does this tool allow for editing of data?
\item Can this tool operate offline?
\item Does this tool have a safe and reliable storage system?
\end{itemize}

\section{References}

Wilkinson, M., Dumontier, M., Aalbersberg, I., Appleton, G., Axton, M., Baak, A., Blomberg, N., Boiten, J., da Silva Santos, L., Bourne, P., Bouwman, J., Brookes, A., Clark, T., Crosas, M., Dillo, I., Dumon, O., Edmunds, S., Evelo, C., Finkers, R., Gonzalez-Beltran, A., Gray, A., Groth, P., Goble, C., Grethe, J., Heringa, J., ’t Hoen, P., Hooft, R., Kuhn, T., Kok, R., Kok, J., Lusher, S., Martone, M., Mons, A., Packer, A., Persson, B., Rocca-Serra, P., Roos, M., van Schaik, R., Sansone, S., Schultes, E., Sengstag, T., Slater, T., Strawn, G., Swertz, M., Thompson, M., van der Lei, J., van Mulligen, E., Velterop, J., Waagmeester, A., Wittenburg, P., Wolstencroft, K., Zhao, J. and Mons, B. (2016). The FAIR Guiding Principles for scientific data management and stewardship. Scientific Data, 3(1).

\end{document}
